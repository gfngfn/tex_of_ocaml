%% Premise:
%%
%% - The interpreter is implemented based on the SECD Machine.
%% - Natural numbers used as indices are represented by (possibly empty) lists of `*`
%%   (see `\secd@AccessNonZeroIndex`).


\def\secd@unique{\secd@unique@dummy}

\def\secd@Error#1{Error: #1}

%% == Instruction tags ==
\def\secd@ACCESS{\secd@HandleACCESS}
\def\secd@CLOSURE{\secd@HandleCLOSURE}
\def\secd@APPLY{\secd@HandleAPPLY}
\def\secd@RETURN{\secd@HandleRETURN}
\def\secd@CONST{\secd@HandleCONST}
\def\secd@PRIM{\secd@HandlePRIM}

%% == Interpreter ==
\def\secd@Run#1#2#3#4{%
  %% #1 : stack
  %% #2 : environment
  %% #3 : code
  %% #4 : dump
  \ifx\secd@unique#3\secd@unique
    %% If code is empty, return the stack
    #1%
  \else
    %% Handle the instruction (by using `\secd@HandleXXXX`)
    #3\secd@End{#1}{#2}{#4}%
  \fi
}
\def\secd@HandleACCESS#1#2\secd@End#3#4#5{%
  %% Handle ACCESS instruction and run the next configuration.
  %% #1 : index for the access
  %% #2 : subsequent code
  %% #3 : stack
  %% #4 : environment
  %% #5 : dump
  \secd@Run{%
  %% next stack:
    %% TODO: should evaluate the first argument before application
    \expandafter\secd@PushToStack\expandafter{\romannumeral-`0\secd@AccessIndex{#1}{#2}}{#3}%
  }{%
  %% next environment:
    #4%
  }{%
  %% next code:
    #2%
  }{%
  %% next dump:
    #5%
  }%
}
\def\secd@AccessIndex#1#2{%
  %% Extract the value from the environment.
  %% This command will be used *under the complete expansion* by `\romannumeral`.
  %% #1 : index for the access
  %% #2 : environment
  \ifx\secd@unique#2\secd@unique
    %% If the given environment is empty, return the ERROR value.
    \secd@ERROR{unbound}%
  \else
    \secd@AccessIndexToNonEmptyEnvironment{#1}#2\secd@End
  \fi
}
\def\secd@AccessIndexToNonEmptyEnvironment#1#2#3\secd@End{%
  %% #1   : index for the access
  %% #2#3 : non-empty environment
  \ifx\secd@unique#1\secd@unique
    %% If the given index is zero, return the first value in the environment.
    #2%
  \else
    %% If the given index is positive, decrement the index,
    %% remove the first element from the environment,
    %% and iteratively call the command.
    \secd@AccessNonZeroIndex#1\secd@End{#3}%
  \fi
}
\def\secd@AccessIndex*#1\secd@End#2{%
  %% #1 : decremented index for the access
  %% #2 : tail environment
  \secd@AccessIndex{#1}{#2}%
}
\def\secd@HandleCLOSURE#1#2\secd@End#3#4#5{%
  %% Handle CLOSURE instruction and run the next configuration.
  %% #1 : code contained in the closure
  %% #2 : subsequent code
  %% #3 : stack
  %% #4 : environment
  %% #5 : dump
  \secd@Run{%
  %% next stack:
    \secd@PushToStack{\secd@CLOS{#1}{#4}}{#3}%
  }{%
  %% next environment:
    #4%
  }{%
  %% next code:
    #2%
  }{%
  %% next dump:
    #5%
  }%
}
\def\secd@HandleAPPLY#1#2\secd@End#3#4#5{%
  %% Handle APPLY instruction and run the next configuration.
  %% #1 : dummy argument
  %% #2 : subsequent code
  %% #3 : stack
  %% #4 : environment
  %% #5 : dump
  \ifx\secd@unique#3\secd@unique
    \secd@Error{stack underflow (1)}%
  \else
    \secd@HandleAPPLY@i#3\secd@End{#2}{#4}{#5}%
  \fi
}
\def\secd@HandleAPPLY@i#1#2\secd@End#3#4#5{%
  %% #1 : the first element of the stack
  %% #2 : the subsequent elements of the stack
  %% #3 : susequent code
  %% #4 : environment
  %% #5 : dump
  \ifx\secd@unique#2\secd@unique
    \secd@Error{stack underflow (2)}%
  \else
    \secd@HandleAPPLY@ii#2\secd@End{#1}{#3}{#4}{#5}%
  \fi
}
\def\secd@HandleAPPLY@ii#1#2\secd@End#3#4#5#6{%
  %% #1 : the second element of the stack (that should be a closure)
  %% #2 : the subsequent elements of the stack
  %% #3 : the first element of the stack
  %% #4 : susequent code
  %% #5 : environment
  %% #6 : dump
  \secd@Run{%
  %% next stack:
    #2%
  }{%
  %% next environment:
    #3#5%
  }{%
  %% next code:
    #4%
  }{%
    #1#6%
  }%
}
\def\secd@HandleRETURN#1#2\secd@End#3#4#5{%
  %% Handle RETURN instruction and run the next configuration.
  %% #1 : dummy argument
  %% #2 : subsequent code
  %% #3 : stack
  %% #4 : environment
  %% #5 : dump
  \ifx\secd@unique#5\secd@unique
    %% If the dump is empty, emit an error.
    \secd@Error{dump underflow}%
  \else
    \secd@HandleRETURN@i{#2}{#3}{#4}#5\secd@End
  \fi
}
\def\secd@HandleRETURN@i#1#2#3\secd@CLOS#4#5#6\secd@End{%
  %% #1 : subsequent code (that will be discarded)
  %% #2 : stack (that should implicitly be non-empty)
  %% #3 : environment (that will be discarded)
  %% \secd@CLOS#4#5 : closure (#4 : code, #5 : environment)
  %% #6 : subsequent element of the dump
  \secd@Run{%
  %% next stack:
    #2%
  }{%
  %% next environment:
    #5%
  }{%
  %% next code:
    #4%
  }{%
  %% next dump:
    #6%
  }%
}
